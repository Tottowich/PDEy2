\section{Exempel}

\subsection{Ekvationer och notation}

För att skriva ekvationer i löpande text använder ni \$\$, t.ex. om ni skriver koden \$f(x)\$ så blir det $f(x)$. Detta bör ni göra varje gång ni vill använda er av variabler och funktioner i löpande text. Större ekvationer, eller ekvationer som kräver speciellt fokus eller numrering bör istället skrivas som nedan,

\begin{equation}
	f(x) = x^2+y_1+y_2.
\end{equation}

\noindent Om ni ej vill ha en viss ekvation numrerad använder ni equation* istället, som nedan.

\begin{equation*}
	f(x) = e^{x_1+y^2}.
\end{equation*}

Ekvationer ska ses som en del i texten och även om de har en egen rad ska det gå att läsa dem som en del av mening. T.ex. meningen ''\textit{Funktionen f definieras som f av x är lika med x i kvadrat, och är en kontinuerlig funktion för alla x.}'' skulle skrivas som:

\bigskip

\noindent Funktionen $f$ definieras som

\begin{equation}
	f(x) = x^2,
\end{equation}
och är en kontinuerlig funktion för alla $x$.

\bigskip

\noindent Notera att det är ett komma efter ekvationen, och om texten efter ekvationen börjar med stor bokstav avslutar ni ekvationen med en punkt.



\noindent För att referera till ekvationer använder ni $\backslash$label$\{\}$ och döper ekvationen till något som ni sedan kan referera till med $\backslash$eqref$\{\}$. Tex. så kan ni referera till ekvationen nedan,

\begin{equation}
	\label{Ekvation1}
    x^2+y^2=1,
\end{equation}

\noindent med referensen: ekvation \eqref{Ekvation1}. Vanligtvis numrerar man endast ekvationer som man har refererat till. Nedan följer olika exempel på ekvationer som kan vara användbara:

\begin{equation}
	y = x_1^2+e^{x}+e^{e^{e^{x}}},
\end{equation}

\begin{equation}
	f(x) = \frac{x+y}{x-y} + \frac{\partial f}{\partial y} + \frac{dy}{dx},
\end{equation}

\begin{equation}
	f(x,y) = ( x+y ) + \left( \frac{x+y}{x-y} \right),
\end{equation}

\begin{equation}
	f(y) = \int_0^\infty x dx,
\end{equation}

\begin{equation}
	f(x) = \sum_{n=1}^{5} x^n,
\end{equation}

\begin{equation}
	f(x,y) = x^2 y^2 \qquad f_1(x,y) = 2xy^2 \qquad f_2(x,y) = 2x^2y,
\end{equation}

\begin{align*}
	f(x) & = x^2 + 2x + 1 - (x+1)^2 \\
    & = (x+1)^2 -(x+1)^2 \\
    & = 0,
\end{align*}

\begin{align}
	& f_1 = x, \qquad f_2 = y, \\
    & f_3 = x^2, \qquad f_4=y^2, \\
    & f_5 = \sin (x), \qquad f_6 = \cos (y),
\end{align}

\[
  f(x,y) = \begin{cases}
      x+y \quad \text{if } x\neq 0 \text{ and } y\neq 0, \\
      x-1 \quad \text{if } x=y=0.
  \end{cases}
\]


\subsection{Bilder, tabeller och listor}

\begin{table}[h!]
	\centering
    \caption{Tabelltext.}
    \begin{tabular}{ c c c } % en bokstav för varje kolumn, c för centrerat innehåll
        \hline
        x & y & z \\
        \hline
        1 & 2 & 3 \\
        4 & 5 & 6 \\
        7 & 8 & 9 \\
    \end{tabular}
    \label{Tabell1}
\end{table}

\noindent Referens: Tabell \ref{Tabell1}.

\begin{figure}[h!]
	\centering
    \includegraphics[scale=0.6]{Bild.png}
    \caption{Bildtext.}
    \label{Bild1}
\end{figure}

\noindent Referens: Figur \ref{Bild1}.

\begin{enumerate}
    \item Första raden.
    \item Andra raden.
    \item Tredje raden.
\end{enumerate}