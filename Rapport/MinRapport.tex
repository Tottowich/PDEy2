
\section{Introduction}
\subsection{Computer lab signification}
The computer lab as a whole revolves around numerical methods used to computed discrete Fourier transforms and extract the coefficients. The assignment consisted of 4 different tasks which tests speed and accuracy.


\subsection{Theory}
All theory used to complete the four assignments are present in the \href{https://www.canvas.umu.se/courses/6692/files/folder/Datorlaboration?preview=1394099}{computer lab description}.
\section{Task}
\subsection{Task 1a}
The first task asks to complete a Discrete Fourier Transform of a signal.
\subsection{Task 2a}
\subsection{Task 2b}
\subsection{Task 3a}
\subsection{Task 3b}


\subsubsection{Task 4a}
The codes where compared side by side on the same computer, while using the command “tic toc” to calculate the time for the computer to run the desired function.

\subsubsection{Task 4b}
The low frequencies were removed from the data by not compressing the data that’s under 10\%  of the highest data given. This by changing the code so that all data has to be checked before being compressed. First the code has to find a Max value measured in Hz. When a Max is found, all data that is under 10\% of Max, gets ignored. The command ifft is used to see if the conjugate of the vector in y is symmetric.  

Detta upprepas tills att approximeringen $p_n = (x_n,y_n)$ har uppnått att: 
\begin{equation*}
    \frac{\partial f}{\partial x}\biggr\rvert_{p_n}<TOL  \;\wedge\;  \frac{\partial f}{\partial y}\biggr\rvert_{p_n}<TOL.
\end{equation*}
Givet någon tolerans, $TOL$, vilket valdes till $10^{-4}$ för att säkerställa en noggrannhet på minst fyra decimaler. Metoden avbryts då antal iterationer överstiger en fixerad begränsning för att undvika överflödig beräkning eller tills toleransen är nådd.
\subsubsection{Sjunkgradient}
Sjunkgradient är en metod som använts för att hitta minimum punkter på den yta beskrivet av funktionen av berget. Sjunkgradienten uppdaterar alla variabler genom att "gå" åt det håll dit gradienten lutar mest nedåt ($-\nabla f$). Generellt kan den n'te iterationen av sjunkgradient beskrivas av
\begin{equation}
    x_i^{(n+1)}=x_i^{(n)}-\alpha\frac{\partial f}{\partial x_i}\biggr\rvert_{\hat{\boldsymbol{x}}^{(n)}}.\\
\end{equation}
Eller på vektorform:
\begin{equation}
    \hat{\boldsymbol{x}}^{(n+1)}=\hat{\boldsymbol{x}}^{(n)}-\alpha\nabla f(\hat{\boldsymbol{x}}^{(n)}).\\
\end{equation}
Där $f$ är någon differentierbar reellvärd funktion med $k\in \mathbb{Z}$ stycken oberoende variabler. $x_i^{(n)}$ är den i'te oberoende variabeln efter n stycken iterationer av sjunkgradient och $\alpha$ är någon liten reellvärd parameter.
\begin{equation*}
    \hat{\boldsymbol{x}}=(x_1,x_2,...,x_k)\in{\rm I\!R^k}\\
\end{equation*}
\begin{equation*}
    f(\hat{\boldsymbol{x}}) = f(x_1,x_2,...,x_k).
\end{equation*}
Processen upprepas till och med varje element av $\nabla f(\hat{\boldsymbol{x}}^{(n)})$ är mindre än någon liten tolerans eller till och med ett maxbelopp av antal iterationer har uppnåtts.

\section{results}
\subsection{Task 4a}
The function mydft compared to the command fft is slower and the time to run the function is significant more affected by the size of the vector. The time for computing the function Mydift gets exponential longer the vector gets exponential bigger. While the value returned by the command fft has a neglibile change in time no matter the size of the vector. 
(sätt in A plot of the running time on the vertical axis and m on the horisontal axis for m = 0, 1, . . . , 11)
\subsection{Task 4b}
When changing the threshold for the frequency, to ignore smaller frequencies, one can observe that the size of the compress file gets significant smaller. This because the function ignores a bit of the data, and the outcome will then be smaller than the data given. The function gets also faster when being able to ignore the small frequencies. By using command ifft, and ignoring small frequencies, the function will be even faster, if the vector in y is conjugate symmetric.

(bilder på a plot of the original data and a plot of the filtred data) 



\subsection{Avslutande kommentar}
Vi har lärt oss applicera våra kunskaper om flervariebelanalys genom att skriva olika funktioner i Matlab. Med hjälp av dessa kunskaper har vi kunnat hitta lokala maximum och minimumpunkter samt implementera en tvådimensionell version av sjunkgradient.
% Create a figure and add a title
\begin{figure}[h]
\centering
\includegraphics[width=0.5\textwidth]{fig1.png}
\caption{A plot of the running time on the vertical axis and m on the horisontal axis for m = 0, 1, . . . , 11}
\label{fig:my_label}
\end{figure}

% Create table of all variables

% Library to use cursive font
\usepackage{mathptmx}


\clearpage